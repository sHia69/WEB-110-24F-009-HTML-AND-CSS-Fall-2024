\documentclass{article}
\usepackage[utf8]{inputenc}
\usepackage{fancyhdr}
\usepackage{titlesec}
\usepackage{enumitem}
\usepackage{listings}
\usepackage{xcolor}

% Header
\pagestyle{fancy}
\fancyhf{}
\fancyhead[L]{Tutorial 1: Getting Started with HTML5}
\fancyhead[R]{Instructor: Mike Girimonte}

% Section formatting
\titleformat{\section}{\normalfont\Large\bfseries}{\thesection}{1em}{}

% Listings settings
\lstset{
  basicstyle=\ttfamily,
  columns=flexible,
  keepspaces=true,
  frame=single,
  backgroundcolor=\color{gray!10},
  keywordstyle=\color{blue},
  commentstyle=\color{green},
  stringstyle=\color{red}
}

\title{Tutorial 1: Getting Started with HTML5}
\author{Hia Al Saleh}
\date{September 5th, 2024}

\begin{document}

\maketitle
\tableofcontents
\newpage 

\section{Objectives}
\begin{itemize}
    \item Explore the history of the web
    \item Create the structure of an HTML document
    \item Insert HTML elements and attributes
    \item Insert metadata into a document
    \item Define a page title
    \item Mark page structures with sectioning elements
    \item Organize page content with grouping elements
    \item Mark content with text-level elements
    \item Insert inline images
    \item Insert symbols based on character codes
    \item Mark content using lists
    \item Create a navigation list
    \item Link to files within a website with hypertext links
    \item Link to e-mail addresses and telephone numbers
\end{itemize}

\section{Exploring the World Wide Web}
\subsection*{1. What is a network?}
A network is a structure in which information and services are shared among devices. A host or a node can be any device that is capable of sending and/or receiving data electronically.

\subsection*{2. What is a server and a client?}
A server is a host that provides information or a service to other devices on the network. A computer or other device that receives a service is called a client. In a client-server network, clients access information provided by one or more users.

\subsection*{3. Types of Networks}
\begin{itemize}
    \item \textbf{Local Area Network (LAN):} A network confined to a small geographic area, such as within a building or department.
    \item \textbf{Wide Area Network (WAN):} A network that covers a wide area, such as several buildings or cities. The largest WAN in existence is the Internet.
\end{itemize}

\section{Introducing HTML}
\subsection*{1. What is HTML?}
A Web page is a text file written in HTML (Hypertext Markup Language). A markup language describes the content and structure of a document by identifying, or tagging, different document elements.

\subsection*{2. The History of HTML}
In the early years of HTML, browser developers were free to define and modify the language as no rules or syntax were defined. The World Wide Web Consortium (W3C) created a set of standards for all browser manufacturers to follow.

\subsection*{3. Differences between HTML and XHTML}
XHTML (Extensible Hypertext Markup Language) is a different version of HTML enforced with a stricter set of standards. HTML5 was developed as the de facto standard for the next generation of HTML.

\section{The Structure of an HTML5 Document}
\subsection*{1. Document Type Declaration}
The first line in an HTML file is the document type declaration, or doctype, that indicates the type of markup language used in the document:
\begin{lstlisting}
<!DOCTYPE html>
\end{lstlisting}

\subsection*{2. Element Tags}
An element tag is the fundamental building block in every HTML document that marks an element in the document. The general syntax of a two-sided element tag is:
\begin{lstlisting}
<element>content</element>
\end{lstlisting}

\subsection*{3. The Element Hierarchy}
An HTML document is divided into two main sections: the head and the body. The head element marks information about the document, while the body element marks the content that will appear in the web page.

\begin{lstlisting}
<!DOCTYPE html>
<html>
<head>
    head content
</head>
<body >
    body content
</body>
</html>
\end{lstlisting}

\subsection*{4. Element Attributes}
Element attributes provide additional information to the browser about the purpose of the element. The general syntax of an element attribute is:
\begin{lstlisting}
<element attr1="value1" attr2="value2">content</element>
\end{lstlisting}

\subsection*{5. Handling White Space}
HTML file documents are composed of text characters and white space. A white-space character is any empty or blank character such as a space, tabs, or a line break. You can use white space to make your file easier to read by separating one code block from another.

\section{Creating the Document Head}
\subsection*{1. Metadata}
The document head contains metadata, which is the content that describes or provides information about how the document should be processed by the browser.

\subsection*{2. Setting the Page Title}
To set the page title, use the following code:
\begin{lstlisting}
<title>Your Page Title</title>
\end{lstlisting}

\subsection*{3. Adding Metadata}
To add metadata, you can use the meta element:
\begin{lstlisting}
<meta name="description" content="A brief description of the page.">
\end{lstlisting}

\subsection*{4. Adding Comments}
A comment is descriptive text that is added to the HTML file but does not appear in the browser window:
\begin{lstlisting}
<!-- This is a comment -->
\end{lstlisting}

\section{Writing the Page Body}
\subsection*{1. Sectioning Elements}
HTML marks the major topical areas of a page using sectioning elements, also referred to as semantic elements. Examples include:
\begin{itemize}
    \item \texttt{<header>}: Represents introductory content or a set of navigational links.
    \item \texttt{<nav>}: Contains navigation links.
    \item \texttt{<section>}: Defines a section in a document.
    \item \texttt{<article>}: Represents a self-contained piece of content.
    \item \texttt{<aside>}: Contains content that is tangentially related to the content around it.
    \item \texttt{<footer>}: Represents the footer for its nearest sectioning content or sectioning root element.
\end{itemize}

\subsection*{2. Grouping Elements}
Grouping elements are used to group content together for styling or layout purposes. The most common grouping elements are:
\begin{itemize}
    \item \texttt{<div>}: A generic container for flow content.
    \item \texttt{<span>}: A generic inline container for phrasing content.
    \item \texttt{<figure>}: Represents self-contained content, often with a caption.
    \item \texttt{<figcaption>}: Represents a caption or legend for a figure.
\end{itemize}

\subsection*{3. Text-Level Elements}
Text-level elements are used to mark content within a block of text. Examples include:
\begin{itemize}
    \item \texttt{<strong>}: Indicates that its contents have strong importance.
    \item \texttt{<em>}: Indicates emphasis that subtly changes the meaning of the sentence.
    \item \texttt{<a>}: Defines a hyperlink.
    \item \texttt{<code>}: Represents a fragment of computer code.
    \item \texttt{<mark>}: Represents text that has been highlighted for reference purposes.
    \item \texttt{<small>}: Represents side comments such as small print.
    \item \texttt{<time>}: Represents a specific period in time.
\end{itemize}

\section{Working with Lists}
\subsection*{1. Creating Lists}
To create an unordered list:
\begin{lstlisting}
<ul>
  <li>Item 1</li>
  <li>Item 2</li>
</ul>
\end{lstlisting}

To create an ordered list:
\begin{lstlisting}
<ol>
  <li>First Item</li>
  <li>Second Item</li>
</ol>
\end{lstlisting}

\subsection*{2. Description Lists}
A description list is created using:
\begin{lstlisting}
<dl>
  <dt>Term</dt>
  <dd>Description of the term.</dd>
</dl>
\end{lstlisting}

\section{Working with Hypertext Links}
\subsection*{1. Creating Hyperlinks}
Hypertext is created by enclosing content within a set of opening and closing \texttt{<a>} tags:
\begin{lstlisting}
<a href="url">Link Text</a>
\end{lstlisting}

\subsection*{2. Linking to E-Mail Addresses}
To create a link to an email address:
\begin{lstlisting}
<a href="mailto:example@example.com">Email Us</a>
\end{lstlisting}

\subsection*{3. Linking to Phone Numbers}
To create a link to a phone number:
\begin{lstlisting}
<a href="tel:+1234567890">Call Us</a>
\end{lstlisting}

\end{document}