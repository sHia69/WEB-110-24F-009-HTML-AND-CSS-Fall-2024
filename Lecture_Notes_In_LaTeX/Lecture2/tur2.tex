\documentclass{article}
\usepackage[utf8]{inputenc}
\usepackage{fancyhdr}
\usepackage{titlesec}
\usepackage{enumitem}
\usepackage{listings}
\usepackage{xcolor}

% Header
\pagestyle{fancy}
\fancyhf{}
\fancyhead[L]{Tutorial 2: Getting Started with CSS}
\fancyhead[R]{Instructor: Mike Girimonte}

% Section formatting
\titleformat{\section}{\normalfont\Large\bfseries}{\thesection}{1em}{}

% Listings settings
\lstset{
  basicstyle=\ttfamily,
  columns=flexible,
  keepspaces=true,
  frame=single,
  backgroundcolor=\color{gray!10},
  keywordstyle=\color{blue},
  commentstyle=\color{green},
  stringstyle=\color{red}
}

\title{Tutorial 2: Getting Started with CSS}
\author{Hia Al Saleh}
\date{September 23rd, 2024}

\begin{document}

\maketitle
\tableofcontents
\newpage 
\section*{Objectives}
\begin{itemize}
    \item Explore the history of CSS
    \item Study different types of style sheets
    \item Explore style precedence and inheritance
    \item Apply color in CSS
    \item Use contextual selectors
    \item Work with attribute selectors
    \item Apply text and font styles
    \item Use a web font
    \item Define list styles
    \item Work with margins and padding space
    \item Use pseudo-classes and pseudo-elements
    \item Insert page content with CSS
\end{itemize}

\section{CSS Styles and Colors}
\subsection{Introducing CSS}
\begin{itemize}
    \item The appearance of the page is determined by one or more style sheets written in the Cascading Style Sheets (CSS) language.
    \item The latest version of the CSS language is CSS3, built upon several modules, each focused on a separate design topic.
\end{itemize}

\subsection{Types of Style Sheets}
\begin{itemize}
    \item \textbf{Browser styles or user agent styles}: Built into the browser.
    \item \textbf{User -defined styles}: Defined by a user based on the configuration settings of the user’s browser.
    \item \textbf{External styles}: Created by a website author, placed within a CSS file, and linked to the page.
    \item \textbf{Embedded styles}: Added to the head of an HTML document.
    \item \textbf{Inline styles}: Added as element attributes within an HTML document and applied to only that particular element.
\end{itemize}

\subsection{Exploring Style Rules}
\begin{itemize}
    \item General syntax of a CSS style rule:
    \begin{lstlisting}
    selector {
        property1: value1;
        property2: value2;
        ...
    }
    \end{lstlisting}
    \item Browser extensions are an extended library of style properties in the browser.
    \item A vendor prefix indicates the browser vendor that created and supports the style property.
\end{itemize}

\subsection{Embedded Style Sheets}
\begin{itemize}
    \item Inserted directly into the HTML file as metadata by adding the following element to the document head:
    \begin{lstlisting}
    <style>
    style rules
    </style>
    \end{lstlisting}
\end{itemize}

\subsection{Inline Styles}
\begin{itemize}
    \item Styles applied directly to specific elements using the following style attribute:
    \begin{lstlisting}
    <element style="property1: value1; property2: value2;">
    content
    </element>
    \end{lstlisting}
\end{itemize}

\subsection{Style Specificity and Precedence}
\begin{itemize}
    \item The more specific style rule has precedence over the more general style rule.
    \item Specificity is an issue when two or more styles conflict.
    \item If two rules have equal specificity and equal importance, the one defined last has precedence.
\end{itemize}

\subsection{Style Inheritance}
\begin{itemize}
    \item Style inheritance is the process in which properties are passed from a parent element to its children.
    \item Example:
    \begin{lstlisting}
    article { color: blue; }
    p { text-align: center; }
    \end{lstlisting}
\end{itemize}

\section{Working with Color in CSS}
\subsection{Color Values}
\begin{itemize}
    \item RGB triplet: The intensity of primary colors expressed as a set of numbers in CSS.
    \begin{lstlisting}
    rgb(red, green, blue)
    \end{lstlisting}
    \item Hexadecimal numbers: A number expressed in the base 16 numbering system.
\end{itemize}

\subsection{HSL Color Values}
\begin{itemize}
    \item Hue: Tint of a color, represented by a direction on a color wheel.
    \item Saturation: Measures the intensity of a color ( ranging from 0\% (no color) to 100\% (full color)).
    \item Lightness: Measures the brightness of a color (ranging from 0\% (black) to 100\% (white)).
\end{itemize}

\subsection{Defining Semi-Opaque Colors}
\begin{itemize}
    \item Opacity defines how solid a color appears, specified using:
    \begin{lstlisting}
    rgba(red, green, blue, opacity)
    hsla(hue, saturation, lightness, opacity)
    \end{lstlisting}
    where opacity ranges from 0 (completely transparent) to 1.0 (completely opaque).
\end{itemize}

\subsection{Setting Text and Background Color}
\begin{itemize}
    \item CSS defines the text and background color for each element on a webpage:
    \begin{lstlisting}
    color: color;
    background-color: color;
    \end{lstlisting}
\end{itemize}

\section{Using Pseudo-Classes and Pseudo-Elements}
\subsection{Pseudo-Classes}
\begin{itemize}
    \item A pseudo-class classifies an element based on its current status, position, or use in the document:
    \begin{lstlisting}
    element: pseudo-class
    \end{lstlisting}
\end{itemize}

\subsection{Pseudo-Elements}
\begin{itemize}
    \item A pseudo-element exists only in the rendered page and can be selected using:
    \begin{lstlisting}
    element::pseudo-element
    \end{lstlisting}
\end{itemize}

\subsection{Generating Content with CSS}
\begin{itemize}
    \item New content can be added before or after an element using:
    \begin{lstlisting}
    element::before { content: text; }
    element::after { content: text; }
    \end{lstlisting}
\end{itemize}

\section{Formatting Lists}
\begin{itemize}
    \item To change the type of list marker or prevent any display of a list marker:
    \begin{lstlisting}
    list-style-type: type;
    \end{lstlisting}
\end{itemize}

\section{Working with Margins and Padding}
\begin{itemize}
    \item To set the width of the padding space:
    \begin{lstlisting}
    padding: size;
    \end{lstlisting}
    \item To set the size of the margin around block-level elements:
    \begin{lstlisting}
    margin: size;
    margin: top right bottom left;
    \end{lstlisting}
\end{itemize}

\end{document}